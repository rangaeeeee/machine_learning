\documentclass{book}
% list of packages 
\usepackage{amsmath}
\usepackage{amssymb}
\usepackage{mathtools}
\usepackage[ampersand]{easylist}
\usepackage[margin=1in]{geometry}

\begin{document}

\title{Machine Learning by Derek Kane}
\author{Rangarajan R}

\tableofcontents
\maketitle

\chapter{Building and Sustaining Predictive Analytics Capabalities}

\section{Predictive Analytics in the Marketplace}
\begin{itemize}
  \item In the 21\textsuperscript{st} century, organizations that do not focus increasing their predictive analytical capabilities will erode their competitive advantage and run the risk of failing.
  \item Regardless of size and market, all business organizations utilize some level of analytics for decision making. The most successful organizations have some degree of predictive analytics incorporated.
\end{itemize}
\section{Introduction to Business Intelligence Systems}

\subsection{What is Business Intelligence ?}
\begin{itemize}
\item Business Intelligence(BI) refers to skills, processes, technologies, applications and practices used to support decision making.
\item BI is a broad category of applications and technologies for gathering, storing, analysing and providing access to data to help clients to make better business decisions.
\item A system that collects, integrates, analyses and presents business information to support better business decision making.
\item BI is an environment in which business users receive information that is reliable, secure, consistent, understandable, easily manipulated and timely ... facilitating more informed decision making.
\end{itemize}

\subsection{Benefits of Bussiness Intelligence}
\begin{itemize}
\item Get the right information to the right people at the right time.
\item Improve Operational efficiency.
\item Eliminate report backlog and delays.
\item Find the root causes and take action.
\item Negotiate better contracts with suppliers and customers.
\item Identity wasted resources and reduce inventory costs.
\item Sell information to customers, partners and suppliers.
\item Leverage your investment in your ERP or data warehouse.
\item Improve startegies with better marketing analysis.
\item Give users the means to make better decisions.
\item Challenge assumptions with factual information.
\end{itemize}

\subsection{Challenges of Building BI Solutions}
\begin{itemize}
\item Data exists in multiple places.
\item Data is not formatted to support complex analysis.
\item Different kind of workers have different data needs.
\item What data should be examined and in what detail?
\item How will users interact with that data?
\end{itemize}

\section{Introduction to Predictive Analytics Technologies}
\subsection{What is Predictive Analytics?}
\begin{itemize}
\item A set of Business Intelligence technologies that uncovers relationships and patterns within large volumes of data that can be used to predict behaviour and events.
\item Predictive Analytics is forward looking, using past events to anticipate the future.
\end{itemize}

\subsection{Predictive Modeling Methods}
\begin{itemize}
\item Analysts build models using different techniques: neural networks, decision trees, linear regression, ARIMA, Naive Bayes,
\item In order to create effective analytic models, the analyst needs to know which models and algorithms to use.
\item Many analytic workbenches now automatically apply multiple models(prediction, classification, segmentation, association detection) to a problem to find the combination that works best.
\item 
\end{itemize}

\chapter{EDA and Model Selection}
\chapter{Regression Analysis and ANOVA Concepts}
\section{Introduction to Regression Analysis}
\subsection{History}
\begin{itemize}
\item The earliest form of regression was the method of least squares, which was published by a French mathematican Adrien-Marie Legendre in 1805 and by German mathematican Guass in 1809.
\item Legendre and Guass both applied the method to the problem of determining, from astronomical observations, the orbits of bodies about the Sun(mostly comes, but also later the then newly discovered minor planets)
\item In the 1950s and 1960s economists used electromechanical desk calculators to calculate regressions. Before 1970, it sometimes took up to 24 hours to receive the result from one regression.
\end{itemize}
\subsection{Introduction}
\begin{itemize}
\item Regression Analysis is the art and science of fitting straight lines to patterns of data.
\item Regression analysis is widely used for prediction and forecasting, where its use has substantial overlap with the field of machine learning.
\item In a linear regression model, the variable of interest is (dependent variable) is predicted from a single or multiple group of variables (independent variable) using a linear mathematical formula.
\item Regression analysis is also used to understand which among the independent variables are related to the dependent variable, and to explore the forms of these relationships.
\item Familiar methods such as linear regression and ordinary least squares regression are parametric, in that regression function is defined in terms of a finite number of unknown parameters that are estimated from the data.
\item Non-parametric regression refers to techniques that allow the regression function to lie in a specified set of functions, which may be infinite-dimensional.
\end{itemize}
\subsection{Business application of Regression analysis}
\begin{itemize}
\item Measuring the impact on a corporation's profits of an increase in profits.
\item Understanding how sensitive a corporation's sales are to changes in advertising expenditures.
\item Seeing how a stock price is affected by changes in interest rates.
\item Calculating price elasticity for goods and services.
\item Ligitation and information discovery.
\item Total Quality Control Analyses.
\item Human Resource and talent evaluation.
\end{itemize}
\begin{itemize}
\item
\end{itemize}
\begin{itemize}
\item
\end{itemize}

\chapter{Decision Trees and Random Forests}
\chapter{Market Basket and Product Recommendation Engines}
\chapter{Cluster Analysis}
\chapter{Artifical Neural Network}
\chapter{Support Vector Machines}
\chapter{Time Series Forecasting}
\chapter{Text Analytics}
\chapter{Ridge Analysis, LASSO and Elastic Net}
\chapter{Hidden Markov Models}
\chapter{Genetic Algorithms}
\chapter{MARS, Logistic Regression and Survival Analysis}
\chapter{Fourier Analysis}
\chapter{Deep Learning and Image Processing}
\chapter{Big Data Fundamentals for Data Scientists}

\end{document}


